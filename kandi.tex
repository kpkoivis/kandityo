
% --- Template for thesis / report with tktltiki2 class ---
% 
% last updated 2013/02/15 for tkltiki2 v1.02

\documentclass[12pt,finnish]{tktltiki2}

% tktltiki2 automatically loads babel, so you can simply
% give the language parameter (e.g. finnish, swedish, english, british) as
% a parameter for the class: \documentclass[finnish]{tktltiki2}.
% The information on title and abstract is generated automatically depending on
% the language, see below if you need to change any of these manually.
% 
% Class options:
% - grading                 -- Print labels for grading information on the front page.
% - disablelastpagecounter  -- Disables the automatic generation of page number information
%                              in the abstract. See also \numberofpagesinformation{} command below.
%
% The class also respects the following options of article class:
%   10pt, 11pt, 12pt, final, draft, oneside, twoside,
%   openright, openany, onecolumn, twocolumn, leqno, fleqn
%
% The default font size is 11pt. The paper size used is A4, other sizes are not supported.
%
% rubber: module pdftex

% --- General packages ---

\usepackage[utf8]{inputenc}
\usepackage[T1]{fontenc}
\usepackage{lmodern}
\usepackage{microtype}
\usepackage{amsfonts,amsmath,amssymb,amsthm,booktabs,color,enumitem,graphicx}
\usepackage[pdftex,hidelinks]{hyperref}

\usepackage{setspace}
\onehalfspacing

\usepackage{rotating}
\usepackage{tikz}
\usepackage{tikz-qtree}
\usetikzlibrary{shapes,snakes}
\usepackage{algpseudocode}
\usepackage{breqn}
\usepackage{array}

% Automatically set the PDF metadata fields
\makeatletter
\AtBeginDocument{\hypersetup{pdftitle = {\@title}, pdfauthor = {\@author}}}
\makeatother

% --- Language-related settings ---
%
% these should be modified according to your language

% babelbib for non-english bibliography using bibtex
\usepackage[fixlanguage]{babelbib}
\selectbiblanguage{finnish}

% add bibliography to the table of contents
\usepackage[nottoc]{tocbibind}
% tocbibind renames the bibliography, use the following to change it back
\settocbibname{Lähteet}

% --- Theorem environment definitions ---

\newtheorem{lau}{Lause}
\newtheorem{lem}[lau]{Lemma}
\newtheorem{kor}[lau]{Korollaari}

\theoremstyle{definition}
\newtheorem{maar}[lau]{Määritelmä}
\newtheorem{ong}{Ongelma}
\newtheorem{alg}[lau]{Algoritmi}
\newtheorem{esim}[lau]{Esimerkki}

\theoremstyle{remark}
\newtheorem*{huom}{Huomautus}

%\newcolumntype{L}{>{\centering\arraybackslash}m{2cm}}
\newcolumntype{L}{>{\arraybackslash}m{2cm}}

% --- tktltiki2 options ---
%
% The following commands define the information used to generate title and
% abstract pages. The following entries should be always specified:

\title{Päätöspuut ja niiden käyttö terveydenhuollossa päätöksenteon tuen menetelmänä}
\author{Kristian Koivisto}
\date{\today}
\level{Kandidaatintutkielma}
\abstract{Tässä työssä perehdytään päätöspuiden perusteisiin, niiden muodostamisessa käytettyihin induktioalgoritmeihin
sekä päätöspuiden sovellukseen terveydenhuollossa. Induktioalgoritmeista ID3:een perehdytään perusteellisemmin.
Päätöspuut ovat tehokkaita luokittimia ja niitä voidaan käyttää päätöksenteon apukeinona tilanteissa, joissa
päätöksenteko on luonteeltaan luokittelua - esimerkiksi lääketieteellisessä diagnostiikassa on usein kyse siitä, mihin
luokkaan tai diagnoosiin potilastapaus kuuluu sen ominaisuuksien perusteella. Erilaisista luokittimista päätöspuut
saattavat soveltua terveydenhuoltoon paremmin kuin muun tyyppiset luokittimet, koska niiden toiminta on
visualisoitavissa ja lyhyen perehtymisen jälkeen on usein suhteellisen helppo kenen tahansa nähdä, mihin päätöspuun
toiminta ja luokittelu perustuu. Tällä on merkitystä, koska terveydenhuollossa erilaisia päätöksentuen apukeinoja on
ollut vaikea istuttaa sairaaloiden ja muiden toimintayksiköiden prosesseihin. Yksi syy tähän on, että terveydenhuollon
ammattilaisten on vaikea luottaa sovelluksiin, joiden tarjoaman tuloksen syitä ja perusteita ei itse pysty verifioimaan.}


% The following can be used to specify keywords and classification of the paper:

\keywords{päätöspuu, päätöspuuinduktioalgoritmi, ID3, CART, terveydenhuolto, päätöksenteon tuki}

% classification according to ACM Computing Classification System (http://www.acm.org/about/class/)
% This is probably mostly relevant for computer scientists
% uncomment the following; contents of \classification will be printed under the abstract with a title
% "ACM Computing Classification System (CCS):"
% \classification{}

% If the automatic page number counting is not working as desired in your case,
% uncomment the following to manually set the number of pages displayed in the abstract page:
%
% \numberofpagesinformation{16 sivua + 10 sivua liitteissä}
%
% If you are not a computer scientist, you will want to uncomment the following by hand and specify
% your department, faculty and subject by hand:
%
% \faculty{Matemaattis-luonnontieteellinen}
% \department{Tietojenkäsittelytieteen laitos}
% \subject{Tietojenkäsittelytiede}
%
% If you are not from the University of Helsinki, then you will most likely want to set these also:
%
% \university{Helsingin Yliopisto}
% \universitylong{HELSINGIN YLIOPISTO --- HELSINGFORS UNIVERSITET --- UNIVERSITY OF HELSINKI} % displayed on the top of the abstract page
% \city{Helsinki}
%


\begin{document}

% --- Front matter ---

\frontmatter      % roman page numbering for front matter

\maketitle        % title page
\makeabstract     % abstract page

\tableofcontents  % table of contents

% --- Main matter ---

\mainmatter       % clear page, start arabic page numbering

\section{Johdanto}
Terveydenhuollossa pyritään resurssit huomioiden takaamaan potilaille parasta mahdollista
hoitoa. Tämä edellyttää laadukasta päätöksentekoa. Usein hyvän päätöksen tekeminen
potilaan hoitoon liittyvässä päätöksentekotilanteessa on kuitenkin haastavaa ja valitettavan
monesti tehty päätös osoittautuu jälkeenpäin epäedulliseksi. Tämän vuoksi terveydenhuollon
ammattilaisille on pyritty kehittämään erilaisia päätöksenteon tuen menetelmiä.
Luonnollisesti tietojenkäsittelytieteen keinoja on valjastettu terveydenhuollon
päätöksenteon laadun parantamiseksi – myös älykkäiden järjestelmien soveltamista tähän
on tutkittu aina tekoälyn alkutaipaleelta 1950-luvulta lähtien. Yksi käyttökelpoiseksi
osoittautunut menetelmä on päätöspuu.

Tässä kirjoituksessa perehdytään päätöspuihin, niiden
muodostamisessa käytettyihin induktioalgoritmeihin sekä muutamiin sovelluksiin
terveydenhuollossa.


\section{Päätöspuut ja niiden käyttö päätöksenteon tuen menetelmänä}
Terveydenhuollon ammattilaiset kohtaavat työssään monenlaisia päätöksentekotilanteita. Mikäli
päätöksenteossa on kyse siitä, mihin luokkaan jokin asia kuuluu siitä tehtävien havaintojen
perusteella, voi päätöspuista olla apua kyseisen päätöksenteon tukimenetelmänä. Tämäntyyppisiä
päätöksentekotilanteita on terveydenhuollossa runsaasti - esimerkiksi menetelmä, jolla
voitaisiin luotettavasti kategorisoida päivystyspoliklinikalle saapuva rintakipupotilas
pienen tai suuren riskin potilaaksi kuolemanvaaran suhteen voisi olla hyödyllinen.
Kyseisen päätöksen tekeminen on harjaantuneellekin lääkärille vaativa tehtävä ja
usein alustavan päätöksen joutuu tekemään aloittelija, jolloin virheellisen päätöksen
mahdollisuus luonnollisesti kasvaa. Tällaisessa asetelmassa luotettavasta päätöksentuen
apukeinosta voisi olla hengenpelastavaakin merkitystä.

Vastaavanlaista luokittelua voidaan aikaansaada monilla muillakin menetelmillä, kuten neuroverkoilla
ja tukivektorikoneilla. Päätöspuu on kuitenkin terveydenhuollon kontekstissa erityisen
mielenkiintoinen vaihtoehto, sillä sen toimintaperiaate on helposti kenen tahansa ymmärrettävissä,
siitä voidaan piirtää visuaalinen esitys ja lyhyen perehtymisen jälkeen on usein suhteellisen helppoa nähdä, miksi
päätöspuu päätyy tiettyyn tulokseen. Tämä on tärkeää, sillä vaikka monien tekoälysovellusten
on osoitettu kykenevän parempaan päätöksentekoon kuin terveydenhuollon ammattilaiset ja näin
niistä saattaisi hyvinkin olla hyötyä, niiden istuttaminen sairaaloiden ja muiden terveydenhuollon
toimintayksiköiden prosesseihin ei ole onnistunut odotetunlaisesti.
Tämän on arveltu johtuvan osittain siitä, että päätöksentekotilanteissa, joihin liittyy epävarmuutta
ja potilaan kannalta mahdollisesti huonoja vaihtoehtoja, terveydenhuollon ammattilaisten on hankala
luottaa laitteeseen, jonka toimintaa on vaikeaa tai mahdotonta ymmärtää \cite{comingOfAgeOfAIInMedicine}. 
Tässä mielessä päätöspuut saattavat olla helpommin
hyväksyttävissä kuin muut luokittimet, kun päätöksentekijä voi itse verifioida, mihin päätöspuun
tulos perustuu. Terveydenhuollossa on myös ennestään totuttu erilaisiin graafisiin päätöksenteon
apuvälineisiin, jolloin päätöspuu ei välttämättä edes olisi niin erilainen kuin jo aiemmin tutuiksi
tulleet menetelmät.
 
Päätöspuu ja sen käyttö päätöksenteon tukemisessa on helposti ymmärrettävissä. Ensinnäkin päätöspuu
on rakenteeltaan nimensä mukaisesti puu ja on sellaisena visualisoitavissa. Sen käyttäminen päätöstä
tehdessä liittyy polun kulkemiseen puun juuresta johonkin sen lehdistä. Puun lehdet edustavat päätöstä
ja sen muut solmut ehtoja, joiden tuloksen perusteella valikoituu yksi solmun lapsipuiden juurista
polun seuraavaksi askeleeksi. Näin kukin polku puun juuresta lehteen koostuu sarjasta ehtoja, joiden
perusteella päädytään yhteen mahdollisista puun tuloksista eli päätöksistä.

Konkreettisena esimerkkinä päätöspuu voisi liittyä edellämainittuun kysymykseen siitä, onko
päivystyspoliklinikalle saapuvalla rintakipupotilaalla suuri kuolemanriski vai ei. Tähän tarkoitukseen
muodostetussa puussa juuressa oleva ehto voisi liittyä potilaan sukupuoleen: jos potilas on miespuolinen,
valitaan juuren vasemmanpuoleinen lapsi, jos taas potilas on naispuolinen, valitaan
juuren oikeanpuoleinen lapsi. Mikäli potilas on mies, saattaisi seuraava ehto liittyä siihen,
onko potilaalla verenpainetauti ja seuraava ehto sen jälkeen taas kolesterolitasoon. Näin
potilaasta tehtävät havainnot johdattavat lopulta puun lehteen eli potilastapauksesta tehtävään
luokitukseen ja sitä myötä päätökseen.

Periaatteessa päätöspuina voitaisiin pitää kaikkia sellaisia puita, joissa edellämainitulla tavalla
päädytään puun lehdessä olevaan päätökseen tiettyjen ehtojen perusteella. Jäljempänä päätöspuilla
tarkoitetaan kuitenkin vain sellaisia puita, jotka on muodostettu niin kutsutuilla induktioalgoritmeilla.
Tällaisille puille on ominaista, että ne ovat luonteeltaan yleistämiseen kykeneviä luokittimia ja
pohjautuvat induktioalgoritmille syötettyihin esimerkkitapauksiin. Induktioalgoritmeihin
perehdytään tarkemmin kappaleessa \ref{sec:induktioalgoritmit}. Yleistämisellä tarkoitetaan sitä,
että päätöspuu kykenee luokittelemaan muitakin tapauksia kuin pelkästään niiden muodostamisessa
käytettyjä esimerkkitapauksia.

Päätöspuiden ymmärrettävyys ja visuaalisuus ovat potentiaalisesti hyödyllisiä sivuominaisuuksia;
induktioalgoritmien tavoitteena on kuitenkin vain puun mahdollisimman hyvä luokituskyky. Puun muilla
ominaisuuksilla ei ole merkitystä. Päätöspuiden soveltamista terveydenhuollon päätöksentekotilanteisiin
tarkastellaan kappaleessa \ref{sec:sovellukset_terveydenhuollossa}. Siinä kontekstissa arvioidaan
päätöspuiden meriittejä myös mainittujen sivuominaisuuksien näkökulmasta, vertailukohtana
terveydenhuollossa tyypillinen, kotikutoinen, kuvassa \ref{fig:Vuokaaviopäätöspuu} esitetty päätöspuu,
jonka muodostamisessa ei ole käytetty mitään erityisempää formalismia. Tämän tyyppinen, vuokaaviomainen
päätöspuu on ilmeisen käyttökelpoinen; induktioalgoritmipäätöspuiden olisi hyvä kyetä samantyyppiseen
intuitiiviseen käytettävyyteen saavuttaakseen hyväksyttävyyttä terveydenhuollon ammattilaisten
keskuudessa.
 
Päätöspuita on sovellettu tiedon louhinnassa ja lisäksi joillakin induktioalgoritmeilla voidaan muodostaa
päätöspuiden lisäksi regressiopuita. Näitä aihepiirejä ei käsitellä tämän enempää.

\begin{figure}
\begin{tikzpicture}[font=\tiny, every tree node/.style={text width=2cm, draw,rectangle},
every leaf node/.style={draw,circle},
   level distance=2cm,sibling distance=0.1cm,
   edge from parent path={(\tikzparentnode) -- (\tikzchildnode)}]
\Tree
[.{Potilaalla krooninen neuropaattinen kiputila?}
    \edge node[auto=right] {Ei};
    [.{Arvioi muita vaihtoehtoja} ]
    \edge node[midway,left] {Kyllä};
    [.{Tavallisia lääkityksiä kokeiltu?} 
       \edge node[midway,left] {Ei};
       [.{Kokeillaan tavallisia lääkityksiä.} ]
       \edge node[midway,right] {Kyllä};
       [.{Tavalliselle lääkitykselle hyvä vaste?}
          \edge node[midway,left] {Ei};
	  [.{Halukas kokeilemaan marijuanaa?}
	     \edge node[midway,left] {Ei};
             [.{Jatketaan tavallisella lääkityksellä.} ]
             \edge node[midway,right] {Kyllä};
             [.{Kuuluuko potilas riskiryhmään (mielenterveys, väärinkäyttäjä)?}
                 \edge node[midway,left] {Ei};
                 [.{Varaa lääkärin arviointikäynti.} ]
                 \edge node[midway,right] {Kyllä};
                 [.{Lähetä psykiatrin arvioon. Psykiatrilta puoltava arvio?} 
		     \edge node[midway,left] {Ei};
                     [.{Jatketaan tavallisella lääkityksellä.} ]
                     \edge node[midway,right] {Kyllä};
                     [.{Varaa lääkärin arviointikäynti.} ] 
                 ]
              ]
              ]
	   \edge node[midway,right] {Kyllä};
	   [.{Jatketaan tavallisella lääkityksellä} ]
	]
    ]
]
\end{tikzpicture}
\caption{Päätöspuu liittyen kysymykseen, hyötyykö potilas lääkärin arviointikäynnistä.}
\label{fig:Vuokaaviopäätöspuu}
\end{figure}

\section{Päätöspuiden induktioalgoritmit}
\label{sec:induktioalgoritmit}
Induktioalgoritmien tehtävä on muodostaa yleistämiseen kykenevä päätöspuu pelkästään niille
syötettyjen, ennalta luokiteltujen esimerkkien perusteella. Kaikki esimerkit oikein
luokittelevan päätöspuun luominen on triviaali tehtävä, lisäksi samasta joukosta esimerkkejä
voidaan muodostaa monenlaisia puita. Haaste onkin muodostaa sellainen puu, joka paitsi luokittelee
esimerkit niin myös mitkä tahansa sille esitetyt tapaukset oikein. Kullakin induktioalgoritmilla
on omanlaisensa strategia suoriutua tästä. Tunnetuin algoritmeista lienee Quinlanin ID3
(Iterative Dichotomiser 3) \cite{quinlan}, joka käsitellään perusteellisesti jäljempänä.
Muita tunnettuja induktioalgoritmeja ovat niin ikään Quinlanin C4.5 \cite{QuinlanC4_5} ja
C5.0 \cite{QuinlanC5_0} sekä Breimanin ym. kehittämä joukko algoritmeja, joita kutsutaan
yhteisellä nimellä CART (engl. classification and regression tree) \cite{CART}.

\subsection{Induktioalgoritmien yhteiset piirteet}
Riippumatta käytetystä algoritmista päätöspuun muodostaminen on aina samanlainen tapahtuma sikäli,
että algoritmin syötteenä on pelkästään joukko esimerkkitapauksia, joiden luokka on tiedossa, ja
tuloksena yleistämiseen pyrkivä luokitin. Kaikkien algoritmien yhteisenä haasteena on esimerkkitapausten
ominaisuuksien perusteella päätellä, mihin luokkaan muutkin mahdollisesti puulle myöhemmin
esiteltävät tapaukset kuuluvat. Tämäntyyppistä päättelyä, jossa esimerkkitapausten perusteella
tehdään oletuksia muistakin mahdollisista tapauksista, kutsutaan induktiiviseksi päättelyksi.
Induktiivinen päättely on luonteeltaan epävarmaa, mistä seuraa, että myös päätöspuiden
aikaansaama luokitus on epävarmaa. Induktioalgoritmien muodostamien päätöspuiden tulosta
täytyykin pitää enemmänkin olettamuksena tai hypoteesina oikeasta luokasta kuin
varmana luokituksena.

Matemaattisessa mielessä päätöspuu toimii kuvauksena eli funktiona kaikkien mahdollisten
tapausten joukosta kaikkien mahdollisten luokkien joukkoon. Kutsuttakoon edellistä joukkoa
joukoksi $S$ ja jälkimmäistä joukoksi $L$. Oletetaan, että jokaisella joukon $S$ alkiolla on
luokka, joka on joukon $L$ alkio. Tällöin jokin tuntematon funktio $f$ kykenee liittämään
jokaisen joukon $S$ alkion oikeaan joukon $L$ alkioon eli täydelliseen luokitteluun.
Funktion $f$ päättely ei yleensä ole mahdollista vaan induktioalgoritmin tuloksena on
hypoteesi $h$, joka on approksimaatio funktiosta $f$. Verrattuna muihin keinoihin approksimoida
$f$ päätöspuuinduktioalgoritmin erikoisuus on päätyminen luokittimeen, joka on rakenteeltaan puu.

Induktioalgoritmit ovat esimerkki ohjatusta oppimisesta; algoritmi pyrkii oppimaan edellisessä
kappaleessa mainitun funktion $f$ sille annettujen ennalta luokiteltujen esimerkkien perusteella. 
Kuten ohjatussa oppimisessa yleensäkin, myös induktioalgoritmien kohdalla käytettävissä olevat
esimerkkitapaukset jaetaan kahteen osajoukoon, opetusjoukkoon ja testijoukkoon. Opetusjoukon
tapauksia käytetään päätöspuun muodostamiseen ja testijoukon tapauksia kyseisen
puun luokituskyvyn selvittämiseen: mitä suurempi osa testijoukon tapauksista luokitellaan oikein,
sitä parempi päätöspuu on tehtävässään. Eri induktioalgoritmit päätyvät erilaisiin puihin vaikka
hyödyntäisivät samaa opetusjoukkoa - vertaamalla puiden luokittelukykyä testijoukon alkioilla
voidaan löytää puista se, joka kussakin päätöksentekotilanteessa toimii parhaiten.

Kukin opetusjoukon tapaus esitellään induktioalgoritmeille kokoelmana attribuutteja, joilla kullakin on arvo.
Eri induktioalgorimit eroavat toisistaan siinä, minkälaisia arvoja attribuutit voivat saada:
jotkut pystyvät käsittelemään reaaliarvoisia attribuutteja, toiset kokonaislukuarvoisia. Kuitenkin esimerkiksi
ID3 pystyy käsittelemään vain sellaisia attribuutteja, joiden arvot tulevat ennalta määritellystä
joukosta. Esimerkki tällaisesta attribuutista voisi olla rintakivun kovuus: ID3:n tapauksessa kovuus voisi
saada vain ennalta määriteltyjä arvoja, kuten ``kova'', ``keskikova'' ja ``lievä'', kehittyneemmässä
algoritmissa arvot voisivat olla jatkuvia esimerkiksi väliltä 0 - 10. Kukin tapaus kuvaillaan
samalla joukolla attribuutteja, kullakin tapauksella on näille omat arvonsa. Tarvittavien attribuuttien määrä
riippuu käsillä olevasta luokitteluongelmasta: joskus luokittelun onnistuminen saattaa vaatia
tapausten kuvailua usealla kymmenellä attribuutilla, toisinaan saatetaan selvitä huomattavasti pienemmällä
määrällä. Voi käydä niinkin, että luokittelu ei onnistu laisinkaan käytettävissä olevilla attribuuteilla.

Valmiissa päätöspuussa opetusjoukon tapausten attribuutteja vastaavat puun ehtosolmujen ehdot ja solmun lapsia
attribuutin mahdolliset arvot. Mikäli rintakipuesimerkin päätöspuu olisi olemassa, siinä saattaisi olla - mutta
ei tarvitse olla - ehtosolmu, jonka ehtona on nimenomaan rintakivun kovuus. Päätöspuuta käytettäessä
rintakipupotilaan rintakivun kovuus pitäisi olla tiedossa ja silloin päätöspuun ehtoon mahdollisuus vastata.
Eteneminen ehtosolmusta eteenpäin määräytyisi juuri tämän rintakipupotilaan rintakivun kovuuden määrittelevän
attribuutin arvon perusteella, jolloin päätöspuun seuraavaksi askeleeksi määräytyisi se ehtosolmun lapsi,
jota vastaa tapauksen arvo. 

\subsection{ID3}
\label{sec:ID3}
Verrattuna sitä edeltäviin induktioalgoritmeihin ID3:ssa oivalluksena on ollut soveltaa informaatioteoriaa
puun ehtosolmujen attribuuttien valinnassa. Algoritmin tiettyjen puutteiden vuoksi siitä on kehitty
paranneltuja versioita (C4.5 ja C5.0). ID3 pystyy luokittelemaan vain kahteen erilliseen luokkaan (+ ja -), lisäksi
attribuuttien arvot voivat olla peräisin vain ennalta määritellystä joukosta vaihtoehtoja. Se ei tue
jatkuva- eikä kokonaislukuarvoisia attribuutteja lainkaan.

ID3 on rekursiivinen algoritmi, joka aloittaa puun muodostamisen sen juuresta. Juuren ollessa kyseessä se pyrkii
valikoimaan koko opetusjoukon $S$ alkioiden attribuuteista sen, jonka suhteen opetusjoukon jakaantumisesta
osajoukkoihin seuraa paras mahdollinen informaatiolisä $IG(S)$ (engl. information gain). Algoritmi laskee
informaatiolisän kaikille attribuuteille: juuren attribuutiksi valikoituu se, jonka informaatiolisä
on suurin. Attribuutin valinnan seurauksena opetusjoukko jakaantuu osajoukkoihin attribuutin mahdollisten arvojen
perusteella ja vastaavasti juuri saa niin monta lapsisolmua kuin attribuutilla on mahdollisia arvoja. Valittu
attribuutti vastaa päätöspuun ehtosolmussa olevaa ehtoa. Jaon jälkeen algoritmi pureutuu jäljellä oleviin
opetusjoukon osajoukkoihin samalla rekursiivisella periaatteella kuin koko opetusjoukon ollessa kyseessä, mutta jo valikoidut
attribuutit jätetään huomioimatta suurinta informaatiolisää laskettaessa eikä niitä enää uudestaan valita
ehtoattribuuteiksi.

Algoritmin rekursio päättyy seuraavissa tilanteissa (pysähtymisehdot): 
\begin{enumerate}
  \item Kaikki jäljellä oleva opetusjoukon osajoukon alkiot kuuluvat samaan luokkaan (+ tai -): muodostetaan puun lehti, jonka luokaksi
  asetetaan kyseisten alkioiden luokka.
  \item Kaikki mahdolliset attribuutit on jo kulutettu loppuun mutta osajoukon alkiot eivät silti kuulu samaan luokkaan (+ tai -): muodostetaan puun lehti,
  jonka luokaksi valikoituu se luokka, johon suurempi osa osajoukon alkioista kuuluu
  \item osajoukko on tyhjä: muodostetaan lehti, jonka luokaksi valitaan lehden vanhemman osajoukon alkioiden yleisin luokka.
\end{enumerate}

ID3-algoritmia käytettäessä käytettävissä olevista esimerkkitapauksista erotetaan satunnaisesti tietty määrä tapauksia; muodostettua
joukkoa kutsutaan ikkunaksi. Jäljelle jääneet tapaukset muodostavat testijoukon. ID3-algoritmi muodostaa ikkunan alkioiden
perusteella päätöspuun, joka aina luokittelee ikkunan tapaukset oikein. Tämän jälkeen loput esimerkkitapaukset eli testijoukko ajetaan
päätöspuun läpi ja mikäli ilmenee, että yksikin testijoukon alkio luokittuu väärin, valitaan satunnainen määrä näitä väärin
luokitettuja tapauksia ja lisätään ne ikkunaan. Iteraatioita jatketaan, kunnes kaikki testijoukonkin alkiot luokittuvat oikein.
Quinlanin artikkelin mukaan \cite{quinlan} tällä tavoin toimien on mahdollista nopeuttaa algoritmin toimintaa. Periaatteessa tällä
strategialla toimien voisi käydä niin, ettei ID3 päätyisi lainkaan sopivaan päätöspuuhun; saman artikkelin
mukaan silloisilla ID3:lle syötetyillä joukoilla ei olisi kertaakaan käynyt niin.

Pseudokoodiesitys ID3-algoritmista on esitettu kuvassa \ref{fig:pseudokoodi}.


\begin{figure}
\begin{algorithmic}
\Function{ID3}{Esimerkit, Attribuutit, Luokat}
\State Muodosta solmu t
\If {Esimerkit-joukon kaikkien alkioiden luokka on  +}
    \State Anna solmulle t leima +
    \State\Return {t}
\EndIf
\If {Esimerkit-joukon kaikkien alkioiden luokka on  -}
    \State Anna solmulle t leima -
    \State\Return {t}
\EndIf
\State Leimaa solmu t Luokat-joukon yleisimmällä arvolla Esimerkit-joukossa  
\If {Attribuutit-joukko on tyhjä}
    \State\Return {t}
\EndIf    
\State Valitse {A*}:ksi {parhaiten luokitteleva Attribuutti Esimerkit-joukossa}
\State Aseta A* t:n ehtoparametriksi
\For{kaikille A*:n mahdollisille arvoille a}
    \State Muodosta t:lle uusi lapsisolmu jota vastaa ehto A* = a
    \State Muodosta joukko EsimerkkiA jossa A* = a
    \If {EsimerkkiA on tyhjä}
        \State Lisää lehtisolmu ja leimaa se Luokat-joukon yleisimmällä arvolla joukossa Esimerkit
    \Else
	\State {Lisää alipuu ID3(EsimerkkiA, Attribuutit \textbackslash {A*}, Luokat)}
    \EndIf
\EndFor
\State\Return {t}
\EndFunction
\end{algorithmic}
\caption{ID3-algoritmin pseudokoodiesitys}
\label{fig:pseudokoodi}
\end{figure}


\subsection{ID3 ja päätöspuun muodostaminen esimerkkidatan pohjalta}

ID3-algoritmin toiminnan havainnollistamiseksi seuraavassa muodostetaan päätöspuu käyttäen taulukossa \ref{table:EsimerkkiData} olevaa esimerkkidataa.
Kyseinen esimerkkidata on kuvitteellinen ja on terveydenhuollon kontekstiin mukailtu versio Quinlanin artikkelissa
\cite{quinlan} olevasta samantyyppisestä esimerkistä. 

Esimerkkidatassa on neljäntoista päivystyspoliklinikalla hoidetun nilkkavammapotilaan tietoja sekä tieto siitä, katsoiko potilaan
päivystyspoliklinikalla hoitanut lääkäri tarpeelliseksi pyytää potilaan nilkasta röntgentutkimusta vai ei (viimeisen
sarakkeen P tai N). Muihin sarakkeisiin on koottu potilaiden tietoja: miltä nilkka on näyttänyt, mikä vammamekanismi on ollut, mikä kivun kovuus
on ollut ja onko potilas kyennyt varaamaan (kävelemään kipeällä nilkalla) vai ei.

Taulukkoa \ref{table:EsimerkkiData} tarkastelemalla herää ajatus, pystyttäisiinkö siinä olevan datan, eli potilaiden ominaisuuksien
ja luokituksen (P tai N) perusteella muodostamaan päätöspuu, jolla voitaisiin ennustaa hoitavan lääkärin päätös tilata potilaan nilkasta
röntgenkuva? Tällainen päätöksenteon apukeino voisi päivystyspoliklinikalla olla hyödyllinen:
päätös kuvaamisen tarpeesta voitaisiin tehdä jo potilaan saapuessa päivystyspoliklinikalle hoitajan tekemän
lyhyen arvion perusteella, jolloin potilas voisi käydä nilkan kuvauksessa ennen lääkärin vastaanotolle saapumista.
Tällöin lääkäri voisi hoitaa potilaan yhden kontaktin perusteella: vastaanotolla olisi potilastapauksen ratkaisemiseksi
kaikki tarpeellinen tieto valmiina. Ilman kuvaamistarpeen ennakointia tarvittaisiin kaksi lääkärin ja potilaan
kohtaamista: ensimmäisessä todetaan, ettei potilastapausta voida ratkaista ilman kuvaamista ja toinen potilastapauksen
lopullista hoitoa varten. Yhden kontakin malli johtaisi potilaan nopeampaan hoitoon ja olisi rajallisten resurssien
tehokasta käyttöä.

Taulukosta nähdään, että kullakin potilastapauksella on neljä attribuuttia ja taulukosta voidaan katsoa kunkin potilaan
saamat attribuuttien arvot. Potilaan 2 attribuutit ja niiden arvot on esitetty taulukossa \ref{table:Nilkkavammapotilas2}.
Taulukosta \ref{table:EsimerkkiData} nähdään myös, että kyseisen potilastapauksen kohdalla ei päädytty kuvaukseen (luokka N).

Kyseinen data sopii ID3:n hyödynnettäviksi: luokitus on jompikumpi kahdesta luokasta (N tai P), luokat ovat erilliset (potilastapaus
ei voi kuulua molempiin luokkiin N ja P samanaikaisesti) ja potilaiden attribuuttien arvot kuuluvat ennaltamääriteltyihin joukkoihin,
jotka on esitetty taulukossa \ref{table:NilkkavammapotilastapautenAttribuutitJaArvot}.

Joukon $S$ entropia $H(S)$ on kyseiseen joukkoon sisältyvän epävarmuuden mitta. Se lasketaan
kaavalla

\[ H(S) = -\sum_{x \in X} p(x)\log_2 p(x) \mbox{,}\]
jossa $X$ on joukon $S$ alkioiden mahdollisten luokkien joukko ja $p(x)$ niiden alkioiden osuus
joukossa $S$ joiden luokka on $x \in X$. Entropian yksikkönä on tapana käyttää bittiä.

ID3-algoritmissa joukko $S$ halutaan jakaa sen attribuutin $A$ suhteen osajoukkoihin
$T_1, T_2, ..., T_t \subset T = S$, joista seuraa suurin informaatiolisä $IG(A,S)$.
Edellä joukot $T_1, T_2, ..., T_t$ ovat erillisiä ja 
$T_1\cup T_2\cup ...\cup T_t = \cup_{t=1}^{n}T_t = S$.
Attribuutilla $A$
on $n$ mahdollista arvoa. Informaatiolisä on joukon $S$ entropian ja osajoukkoihin jaosta seuraavan
entropian erotus, joka lasketaan kaavalla

\[ IG(A,S) = H(S) -\sum_{t = 1}^{n} p(T_t)H(T_t) \mbox{,}\]
jossa $p(T_t)$ on joukon $T_t$ alkioiden osuus kaikista joukon $S$ alkioista ja $H(T_t)$ joukon $T_t$
entropia. 

Lasketaan nilkkavammapotilasdatan entropia. Nilkkavammapotilaat ovat jakautuneet kahteen luokkaan N ja P.
Luokkaan N kuuluu 5 tapausta ja luokkaan P kuuluu 9 tapausta. Yhteensä tapauksia on siis neljätoista. Näiden
tietojen perusteella nilkkavammapotilasjoukon $S$ entropia on

\[ H(S) = -\left(\frac{5}{14}\log_2\left(\frac{5}{14}\right) + \frac{9}{14}\log_2\left(\frac{9}{14}\right)\right) \approx 0.940 \mbox{ bittiä.} \] 

Lasketaan seuraavaksi, minkä attribuutin perusteella koko nilkkavammapotilaiden joukko kannattaa jakaa osajoukkoihin,
eli minkä attribuutin suhteen jakamisesta seuraa suurin informaatiolisä. Aloitetaan attribuutista \textit{Ulkonäkö},
joka taulukon \ref{table:NilkkavammapotilastapautenAttribuutitJaArvot} mukaisesti
voi saada arvot \textit{Normaali, Turvonnut} tai \textit{Virheasento}. Informaatiolisän $IG(A_{Ulkonäkö},S)$ laskemiseksi
tarvitaan $p(T_{Normaali})$, $p(T_{Turvonnut})$ ja $p(T_{Virheasento})$, eli attribuutin \textit{Ulkonäkö} saamien arvojen
osuus kaikista joukon $S$ alkioista. Taulukosta \ref{table:EsimerkkiData} voidaan laskea, että

\begin{equation*}
\begin{split}
p(T_{Normaali}) &= \frac{5}{14} \\
p(T_{Turvonnut}) &= \frac{5}{14} \\
p(T_{Virheasento}) &= \frac{4}{14} \mbox{.}
\end{split}
\end{equation*}

%\[ p(T_{Normaali}) = \frac{5}{14} \]
%\[ p(T_{Turvonnut}) = \frac{5}{14} \]
%\[ p(T_{Virheasento}) = \frac{4}{14} \mbox{.} \]

Lisäksi tarvitaan joukkojen $T_{Normaali}$, $T_{Turvonnut}$ ja $T_{Virheasento}$ entropiat $H(T_{Normaali})$, $H(T_{Turvonnut})$
ja $H(T_{Virheasento})$. Kaava on luonnollisesti sama kuin edellä, jossa laskettiin koko joukon $S$ entropia $H(S)$, mutta
osajoukkojen $T_{Normaali}$, $T_{Turvonnut}$ ja $T_{Virheasento}$ entropioita laskettaessa $p(x)$
tarkoittaa luokkien N ja P osuutta kussakin osajoukossa $T_{Normaali}$, $T_{Turvonnut}$ ja $T_{Virheasento}$.
Entropiat ovat
\begin{equation*}
\begin{split}
H(T_{Normaali}) &= -\left(\frac{3}{5}\log_2\left(\frac{3}{5}\right) + \frac{2}{5}\log_2\left(\frac{2}{5}\right)\right) \approx 0.971 \mbox{ bittiä,} \\
H(T_{Turvonnut}) &= -\left(\frac{3}{5}\log_2\left(\frac{3}{5}\right) + \frac{2}{5}\log_2\left(\frac{2}{5}\right)\right) \approx 0.971 \mbox{ bittiä ja} \\ 
H(T_{Virheasento}) &= -\left(\frac{4}{4}\log_2\left(\frac{4}{4}\right) + \frac{0}{4}\log_2\left(\frac{0}{4}\right)\right) = 0 \mbox{ bittiä.}
\end{split}
\end{equation*}
Viimeistä riviä laskettaessa törmätään lausekkeeseen $0\log_20$, jolla ei ole määriteltyä arvoa. Informaatioteoriassa tämä saa yleisen
käytännön mukaisesti arvon 0. Se, että tästä seuraa $H(T_{Virheasento}) = 0$, on helposti ymmärrettävissä, koska kyseisessä tapauksessa
kaikki alkiot kuuluvat samaan luokkaan P ja näin entropia määritelmän mukaan on $0$.

Nyt on kaikki tarvittava informaatiolisän $IG(A_{Ulkonäkö},S)$ laskemiseksi:
\begin{equation*}
\begin{split}
IG(A_{Ulkonäkö},S) &= H(S) -\sum_{t = 1}^{n} p(T_t)H(T_t) \\
%                  &= H(S) -\Big( p(T_{Normaali})H(T_{Normaali}) ) + p(T_{Turvonnut})H(T_{Turvonnut}) + p(T_{Virheasento})H(T_{Virheasento}) \Big) \\
		   &\approx 0.940 - \left( \frac{5}{14}\cdot0.971 + \frac{5}{14}\cdot0.971  + \frac{4}{14}\cdot0 \right) \\
                   &\approx 0.246.
\end{split}
\end{equation*}

Samalla tavalla on mahdollista laskea myös muidenkin attribuuttien suhteen jakamisesta seuraavat informaatiolisät:
\begin{equation*}
\begin{split}
IG(A_{Vammamekanismi},S) &\approx 0.029 \mbox{,}\\
IG(A_{Kipu},S) &\approx 0.151 \mbox{ ja}\\
IG(A_{Varaaminen},S) &\approx 0.048 \mbox{.}
\end{split}
\end{equation*}
Koska informaatiolisä $IG(A_{Ulkonäkö},S)$ on suurin, valitsee ID3 ulkonäköattribuutin päätöspuun juuren ehtosolmun attribuutiksi.
Tässä vaiheessa aikaansaatu päätöspuu on esitetty kuvassa \ref{fig:NilkkavammaPaatospuuJuuri}. Juuren attribuutin valitsemisen
jälkeen ID3 siirtyy rekursiivisesti rakentamaan juuren lapsipuita. Lapsipuita on kolme, koska juurisolmun ehtoattribuutilla
on kolme mahdollista arvoa. Lapsipuiden juuria valittaessa ID3 ei enää ota huomioon koko opetusjoukkoa $S$ vaan sen
erillisiä osajoukkoja $T_{Normaali}$, $T_{Turvonnut}$ ja $T_{Virheasento}$, yksi kutakin lapsipuuta kohden. Tällä tavoin
rekursiivisesti edeten ID3 rakentaa päätöspuun pysähtyen kunkin lapsipuun kohdalla kappaleessa \ref{sec:ID3}
mainittuihin pysähtymisehtoihin. Joukot $T_{Normaali}$, $T_{Turvonnut}$ ja $T_{Virheasento}$ on esitetty taulukoissa
\ref{table:OsajoukkoTNormaali}, \ref{table:OsajoukkoTTurvonnut} ja \ref{table:OsajoukkoTVirheasento}.
Huomionarvoista on, että kun ulkonäköattribuutti $A_{Ulkonäkö}$ on käsitelty ja kiinnitetty puun juureen, sitä ei enää
uudestaan huomioida juuren lapsipuita rakennettaessa. Yleensäkin attribuutit voivat esiintyä puun poluissa vain kertaalleen -
toisaalta sama attribuutti voi esiintyä useassa polussa eri puolilla päätöspuuta.
ID3:n aikaansaama päätöspuu on esitetty kuvassa \ref{fig:NilkkavammaPaatospuu}.


\begin{table}
\centering
    \begin{tabular}{ | l | L | L | L | L | p{2cm} |} \hline
    \textbf{Nro} & \textbf{Ulkonäkö} & \textbf{Vamma- mekanismi} & \textbf{Kipu} & \textbf{Varaa- minen} & \textbf{Kuvataan} \\ \hline
    1 & Normaali & Nyrjähdys & Ei-Kova & Ei & N \\ \hline
    2 & Normaali & Nyrjähdys & Ei-kova & Kyllä & N \\ \hline
    3 & Virheasento & Nyrjähdys & Ei-kova & Ei & P \\ \hline
    4 & Turvonnut & Putoaminen & Ei-kova & Ei & P \\ \hline
    5 & Turvonnut & Muu & Kova & Ei & P \\ \hline
    6 & Turvonnut & Muu & Kova & Kyllä & N \\ \hline
    7 & Virheasento & Muu & Kova & Kyllä & P \\ \hline
    8 & Normaali & Putoaminen & Ei-kova & Ei & N \\ \hline
    9 & Normaali & Muu & Kova & Ei & P \\ \hline
    10 & Turvonnut & Putoaminen & Kova & Ei & P \\ \hline
    11 & Normaali & Putoaminen & Kova & Kyllä & P \\ \hline
    12 & Virheasento & Putoaminen & Ei-kova & Kyllä & P \\ \hline
    13 & Virheasento & Nyrjähdys & Kova & Ei & P \\ \hline
    14 & Turvonnut & Putoaminen & Ei-kova & Kyllä & N \\ \hline
    \end{tabular}
\caption{Esimerkkidata, joka koostu neljäntoista nilkkavammapotilaan tiedoista.}
\label{table:EsimerkkiData}
\end{table}

\begin{table}
\centering
\begin{tabular}{ | l | l | l | l | l | p{2cm} |} \hline
    \textbf{Attribuutti} & \textbf{Attribuutin arvo} \\ \hline
    Ulkonäkö & Normaali \\ \hline
    Vammamekanismi & Nyrjähdys \\ \hline
    Kipu & Ei-kova \\ \hline
    Varaaminen & Kyllä \\ \hline
\end{tabular}  
\caption{Nilkkavammapotilaan numero 2 attribuuttien arvot}
\label{table:Nilkkavammapotilas2}
\end{table}

\begin{table}
\centering
\begin{tabular}[H]{ | l | l | l | l | l | p{2cm} |} \hline
    \textbf{Attribuutti} & \textbf{Attribuutin mahdolliset arvot} \\ \hline
    Ulkonäkö & Normaali, Turvonnut, Virheasento \\ \hline
    Vammamekanismi & Nyrjähdys, Putoaminen, Muu \\ \hline
    Kipu & Kova, Ei-kova \\ \hline
    Varaaminen & Kyllä, Ei \\ \hline
\end{tabular} 
\caption{Nilkkavammapotilastapausten mahdolliset attribuutit ja niiden arvot.}
\label{table:NilkkavammapotilastapautenAttribuutitJaArvot}
\end{table}

\begin{figure}
\centering
\begin{tikzpicture}[font=\tiny, every tree node/.style={text width=2cm, draw,rectangle},
every leaf node/.style={draw,circle},
   level distance=2cm,sibling distance=0.1cm,
   edge from parent path={(\tikzparentnode) -- (\tikzchildnode)}]
\Tree
[.{Ulkonäkö?}
    \edge node[auto=right] {Normaali};
      \node[draw=none,rectangle] {[lapsipuu $T_{Normaali}$]};
    \edge node[auto=left] {Virheasento};
      \node[draw=none,rectangle] {[lapsipuu $T_{Virheasento}$]};
    \edge node[auto=left] {Turvonnut};
      \node[draw=none,rectangle] {[lapsipuu $T_{Turvonnut}$]};
]
\end{tikzpicture}
\caption{ID3 on valinnut juurisolmun attribuutin. Seuraavaksi algoritmi alkaa työstämään juuren lapsia.
Lapsipuut muodostetaan joukkoihin $T_{Normaali}$, $T_{Turvonnut}$ ja $T_{Virheasento}$ perustuen.}
\label{fig:NilkkavammaPaatospuuJuuri}
\end{figure}



\begin{table}
\centering
    \begin{tabular}{ | l | l | l | l | l | p{2cm} |} \hline
    \textbf{Nro} & \textbf{Vammamekanismi} & \textbf{Kipu} & \textbf{Varaaminen} & \textbf{Kuvataan} \\ \hline
    1 & Nyrjähdys & Ei-kova & Ei & N \\ \hline
    2 & Nyrjähdys & Ei-kova & Kyllä & N \\ \hline
    3 & Putoaminen & Ei-kova & Ei & N \\ \hline
    4 & Muu & Kova & Ei & P \\ \hline
    5 & Putoaminen & Kova & Kyllä & P \\ \hline
    \end{tabular}
\caption{Opetusjoukon osajoukko $T_{Normaali}$.}
\label{table:OsajoukkoTNormaali}
\end{table}

\begin{table}
\centering
    \begin{tabular}{ | l | l | l | l | l | p{2cm} |} \hline
    \textbf{Nro} & \textbf{Vammamekanismi} & \textbf{Kipu} & \textbf{Varaaminen} & \textbf{Kuvataan} \\ \hline
    1 & Putoaminen & Ei-kova & Ei & P \\ \hline
    2 & Muu & Kova & Ei & P \\ \hline
    3 & Muu & Kova & Kyllä & N \\ \hline
    4 & Putoaminen & Kova & Ei & P \\ \hline
    5 & Putoaminen & Ei-kova & Kyllä & N \\ \hline
    \end{tabular}
\caption{Opetusjoukon osajoukko $T_{Turvonnnut}$.}
\label{table:OsajoukkoTTurvonnut}
\end{table}

\begin{table}
\centering
    \begin{tabular}{ | l | l | l | l | l | p{2cm} |} \hline
    \textbf{Nro} & \textbf{Vammamekanismi} & \textbf{Kipu} & \textbf{Varaaminen} & \textbf{Kuvataan} \\ \hline
    1 & Nyrjähdys & Ei-kova & Ei & P \\ \hline
    2 & Muu & Kova & Kyllä & P \\ \hline
    3 & Putoaminen & Ei-kova & Kyllä & P \\ \hline
    4 & Nyrjähdys & Kova & Ei & P \\ \hline
    \end{tabular}
\caption{Opetusjoukon osajoukko $T_{Virheasento}$.}
\label{table:OsajoukkoTVirheasento}
\end{table}

\begin{figure}
\centering
\begin{tikzpicture}[font=\tiny, every tree node/.style={align=center, text width=2cm, draw,rectangle},
   every leaf node/.style={align=center, text width=0.55cm, draw, ellipse},
   level distance=1.5cm,sibling distance=2cm,
   edge from parent path={(\tikzparentnode) -- (\tikzchildnode)}]
\Tree
[.{Ulkonäkö?}
    \edge node[auto=right] {Normaali};
    [.{Kipu?}
      \edge node[auto=right] {Kova};
	[{P} ]
      \edge node[auto=left] {Ei-kova};
	[{N} ]
    ]
    \edge node[auto=right] {Virheasento};
    [{P} ]
    \edge node[auto=left] {Turvonnut};
    [.{Varaaminen?}
      \edge node[auto=right] {Ei};
	[{P} ]
      \edge node[auto=left] {Kyllä};
	[{N} ]
    ]
]
\end{tikzpicture}
\caption{ID3:n aikaansaama päätöspuu taulukon \ref{table:EsimerkkiData} nilkkavammadatasta.}
\label{fig:NilkkavammaPaatospuu}
\end{figure}


\subsection{CART}
CART \cite{CART} on ID3:n ja sen jälkeläisten C4.5:n ja C5.0:n tavoin keskeinen päätöspuuinduktioalgoritmi. Sen toiminta muistuttaa ID3-algoritmia sikäli, että
samalla tavoin päätöspuun muodostaminen aloitetaan puun juuresta ja rekursiivisesti jatkuu kunnes puu on
muodostettu. Oleellinen ongelma puuta muodostettaessa sekä CART:ssa että ID3:ssa on päättää puun ehtosolmujen valinnasta,
jonka seurauksena puu saa muotonsa. Tämäntyyppisellä strategialla toimivia algoritmeja on kutsuttu yhteisellä nimikkeellä TDIDT, joka on
kirjainlyhenne englanninkielisistä nimestä ``top-down induction of decision trees'' \cite{quinlan}.

CART:n toiminta poikkeaa ID3:n toiminnasta monin tavoin ja on sitä huomattavasti monipuolisempi algoritmi.
Se muodostaa aina binääripuita eikä ehtosolmujen attribuuttien valinnassa sovelleta yleensä informaatioteoriaa vaan
kussakin solmussa pyritään aikaansaamaan sellainen joukon jako, jonka seurauksena muodostuvat kaksi osajoukkoa ovat alkuperäistä
joukkoa ``puhtaampia''. Käytettävä epäpuhtauden mitta riippuu tilanteesta: ID3:n tavoin luokittimena käytettäessä
epäpuhtauden mittana voidaan käyttää ns. gini-indeksiä tai ns. twoing-indeksiä. Molemmissa periaate on sama:
joukon epäpuhtaus vähenee kun sen homogeenisuus kasvaa. Epäpuhtaus saa minimiarvon, kun solmun kaikki tapaukset kuuluvat samaan
luokkaan ja maksimiarvon, kun solmun tapaukset jakautuvat kaikkiin luokkiin tasaisesti.

CART on ID3:a joustavampi attribuuttien ominaisuuksien suhteen. Siinä missä ID3 sallii attribuuttien
arvoiksi vain ennalta määriteltyjä, CART osaa ottaa huomioon myös näiden arvojen mahdollisen
ordinaalisuuden ja toisaalta jatkuva-arvoisetkin attribuutit sallitaan. Koska CART muodostaa
binääripuita, kussakin ehtosolmussa otetaan huomioon vain kaksi attribuuttien mahdollista arvoa. Toisaalta
attribuutilla voi olla enemmän kuin kaksi mahdollista arvoa, joten sama attribuutti saattaa esiintyä puun
poluilla useaan kertaan - ID3:ssa attribuutin kaikki arvot käsitellään samassa solmussa ja solmu
saa aina yhtä monta lasta kuin attribuutilla on mahdollisia arvoja.

Luokittelun lisäksi CART kykenee nimensä mukaisesti muodostamaan myös regressiopuita, joissa pyrkimyksenä
ei ole luokittelu kategorisiin luokkiin vaan estimoida jatkuva-arvoisen funktion riippuvuus
attribuuttien arvoista. 

CART-algoritmin kokonaisvaltainen käsittely jää tämän työn ulkopuolelle. Esitellään kuitenkin
gini-indeksi ja ``epäpuhtauden'' merkitys ehtosolmun valinnassa, jotta ero ID3:n
informaatioteoriaan perustuvaan menetelmään olisi arvioitavissa tämän työn puitteissa.

Gini-indeksi joukolle $T$, jossa on $n$ luokkaa, määritellään seuraavasti:
\begin{equation*}
gini(T) = 1 -\sum_{j=1}^{n} p_j^2 \mbox{,}
\end{equation*}
jossa $p_j(t)=\frac{n_j(t)}{n(t)}$ on luokan $j$ osuus solmun $t$ alkioista. Jaetaan joukko $T$ kahteen osajoukkoon
$T_1$ ja $T_2$, joiden alkioiden lukumäärät ovat $m_1$ ja $m_2$. Jaon gini-indeksi määritellään seuraavasti:
\begin{equation*}
gini_{jako}(T) = \frac{m_1}{m_1+m_2}gini(T_1) + \frac{m_2}{m_1+m_2}gini(T_2) \mbox{.}
\end{equation*}
Jos solmussa $t$ tehtäisiin jako $s$ seuraisi tästä gini-indexin koheneminen (epäpuhtuden vähentyminen), joka voidaan
laskea koko joukon $T$ gini-indexin ja jaosta seuraavan gini-indexin erotuksena:
\begin{equation*}
\Delta(s,t) = gini(T) - gini_{jako}(T) \mbox{.}
\end{equation*}
Jaolla $s$ tarkoitetaan yhtä niistä mahdollisista jaoista, joita voidaan tehdä solmussa $t$.
CART-algoritmi valitsee solmun $t$ jaoksi sen jaon $s$, josta seuraa suurin $\Delta(s,t)$.
Valinnan jälkeen ID3-algoritmin tavoin CART jatkaa tämän jälkeen rekursiivisesti solmun $t$
lapsipuiden solmujen järjestyksen selvittämistä, josta seuraa lopulta kokonainen päätöspuu.


\section{Päätöspuiden sovelluksia terveydenhuollossa}
\label{sec:sovellukset_terveydenhuollossa}
Esitellään kolme kirjallisuudessa kuvattua päätöspuiden sovellusta terveydenhuollon kontekstissa.

\subsection{Aivoverenkiertopotilaiden ennustaminen}
Duen-Yian ym. \cite{predictiveModelCerebrovascularDisease} vertasivat kolmea eri menetelmää aivoverisuonisairauksien ennustamiseksi potilaiden ominaisuuksien
perusteella. Parhaimmaksi osoittautui päätöspuihin perustuva
luokitin. Tutkimuksessa käytettiin C4.5-algoritmia, joka on Quinlanin ID3:n pohjalta kehittämä induktioalgoritmi.

Aivoverisuonisairaudet ovat joukko sairauksia, joihin liittyy paljon kuolleisuutta ja elämänlaadun heikentymistä
muun muassa vammautumisen ja dementoitumisen vuoksi. Keskeinen aivoverisuonisairaus on aivoinfarkti, joka
voi johtaa pahimmassa tapauksessa kuolemaan mutta lievänäkin erilaisiin neurologisiin vammoihin, kuten
halvaantumiseen ja liikunta- ja puhekyvyn menettämiseen. Maailmanlaajuisesti aivoverenkierron häiriöt ovat toiseksi yleisin
kuolinsyy \cite{burdenOfDisease}. Niiden osuus kaikista kuolemista on noin 10\%. Vuoden 2002 tilastojen mukaan
\cite{aivohalvaustenIlmaantuvuusSuomessa19912002} Suomessa sairastuu vuosittain ensimmäiseen aivohalvaukseen
\begin{itemize}
  \item 25–74-vuotiaista 252/100~000 miestä ja 137/100~000 naista,
  \item 75–84-vuotiaista 1~747/100~000 miestä ja 1~586/100~000 naista,
  \item yli 85-vuotiaista 3~013/100~000 miestä ja 3~029/100~000 naista.
\end{itemize}
Aivoinfarktin kansantaloudellinen merkitys on valtava: elinikäisiksi terveydenhuollon kustannuksiksi
Suomessa on arvioitu 80 000 euroa ja vuotuisiksi valtakunnallisiksi kustannuksiksi 1,1 miljardia euroa
\cite{aivoinfarktinKustannuksetTHL}.

Aivoverenkierron häiriöiden kansantaloudellisten ja potilaille niistä koituvien haittojen vuoksi
menetelmä, jolla voitaisiin ennustaa yksittäisen potilaan kohdalla riski saada kyseisiä häiriöitä,
olisi luonnollisesti käyttökelpoinen. Riskipotilaisiin voitaisiin kohdistaa ennaltaehkäiseviä
interventioita ja näin mahdollisesti estää näiden potilaiden kohdalla aivoverenkiertohäiriön
toteutuminen. Edellä mainitussa artikkelissa \cite{predictiveModelCerebrovascularDisease}
päädyttiin 493 potilaan tiedoista koostuvassa materiaalissa päätöspuuhun, jossa 29 attribuutin joukosta
valikoituneiden yhdeksän attribuutin perusteella voitiin ennustaa potilaan kuuluminen
aivoverenkiertohäiriöpotilaiden joukkoon. Päätöspuu on esitetty kuvassa \ref{fig:AVHpäätöspuu}.
Puun lehdissä sulkeissa tapausten lukumäärät. Symbolien merkitykset on esitetty taulukossa 
\ref{table:AVHpäätöspuuSelitteet}.


\begin{table}
\centering
    \begin{tabular}{ | l | p{5cm} |} \hline
    \textbf{Symboli} & \textbf{Merkitys} \\ \hline
    Sick & Onko sairautta? \\ \hline
    dm(D) & Onko sokeritautia? \\ \hline
    hp(B) & Onko verenpainetautia? \\ \hline
    lip(B) & Onko hyperlipidemiaa? \\ \hline
    hd(H) & Onko sepelvaltimotautia? \\ \hline
    arr(H) & Onko rytmihäiriöitä? \\ \hline
    mi(H) & Onko ollut sydäninfarktia? \\ \hline
    car(H) & Onko ollut kardiogeenistä shokkia? \\ \hline
    bmi & Painoindeksi? \\ \hline
    N & Ei ole sairautta \\ \hline
    SM & Aivoverenkiertohäiriö ja kaksi muuta sairautta tai sokeritauti ja yksi muu sairaus BH:n listasta \\ \hline
    DM & Aivoverenkiertohäiriö ja sokeritauti \\ \hline
    BH & Aivoverenkiertohäiriö ja yksi muu sairaus (verenpainetauti, reuma, hyperlipidemia, sepelvaltimotauti, rytmihäiriö, sydämen vajaatoiminta, sydäninfarkti) \\ \hline
    CD & Vain aivoverenkierron häiriö, ei muita sairauksia \\ \hline
    \end{tabular}
\caption{Aivoverenkiertopäätöspuun symbolien selitykset.}
\label{table:AVHpäätöspuuSelitteet}
\end{table}

\begin{sidewaysfigure}
\begin{tikzpicture}[font=\fontsize{4.5}{1em}\selectfont, 
every tree node/.style={text width=0.5cm, draw, rectangle},
  every leaf node/.style={text width=0.55cm, draw, ellipse},
  level distance=1.0cm,sibling distance=0.1cm,
  edge from parent path={(\tikzparentnode) -- (\tikzchildnode)}]
\Tree
[.{Sick}
    \edge node[auto=right] {N};
    [{N(109)} ]
    \edge node[auto=left] {Y};
    [.{dm(D)} 
       \edge node[midway,left] {Y};
       [.{hp(B)} 
          \edge node[midway,left] {Y};
          [{SM(51)} ]
          \edge node[midway,right] {N};
          [.{lip(B)}
             \edge node[midway,left] {Y};
             [{SM(7)} ]
             \edge node[midway,right] {N};
             [.{hd(H)} 
                \edge node[midway,left] {Y};
                [{SM(2)} ]
                \edge node[midway,right] {N};
                [.{arr(H)} 
                   \edge node[midway,left] {Y};
                   [{SM(2)} ]
                   \edge node[midway,right] {N};
                   [.{lib(B)} 
                      \edge node[midway,left] {Y};
                      [{SM(3)} ]
                      \edge node[midway,right] {N};
                      [{BH(23)} ]
                   ] 
                ] 
             ]
          ]
       ]
       \edge node[midway,right] {N};
       [.{mi(H)}
       	   \edge node[midway,left] {Y};
	   [.{hp(B)} 
	      \edge node[midway,left] {Y};
              [{SM(5)} ]
              \edge node[midway,right] {N};
              [{BH(6/1)} ] 
	   ]
          \edge node[midway,left] {N};
	  [.{car(H)}
	     \edge node[midway,left] {Y};
             [.{hp(B)}
                 \edge node[midway,left] {Y};
                 [{SM(9)} ]
                 \edge node[midway,right] {N};
                 [.{lip(B)} 
		     \edge node[midway,left] {Y};
                     [{SM(2)} ]
                     \edge node[midway,right] {N};
                     [{BH(13)} ] 
                 ]
              ]
	     \edge node[midway,right] {N};
             [.{arr(H)} 
                \edge node[midway,left] {Y};
	        [.{hp(B)} 
	           \edge node[midway,left] {Y};
                   [{SM(11)} ]
                   \edge node[midway,right] {N};
                   [.{lip(B)} 
                      \edge node[midway,left] {Y};
                      [{SM(3)} ]
                      \edge node[midway,right] {N};
                      [{BH(23)} ] 
                   ] 
	        ]
	        \edge node[midway,right] {N};
	        [.{hd(H)} 
	           \edge node[midway,left] {Y};         
                   [.{hp(B)} 
                      \edge node[midway,left] {Y};
                      [.{bmi} 
                         \edge node[midway,left] {>26.8};
                         [{SM(1)} ]
                         \edge node[midway,right] {=<26.8};
                         [{SM(1)} ] 
                      ]
                      \edge node[midway,right] {N};
                      [{BH(2)} ] 
                   ] 
                   \edge node[midway,right] {N};
                   [{CD(175)} ] 
	        ]
             ]
           ]
	]
    ]
]
\end{tikzpicture}
\label{fig:AVHpäätöspuu}
\caption{Päätöspuu aivoverenkiertohäiriön ennustamiseen.}
\end{sidewaysfigure}

\subsection{Masennuksen ennustaminen}
Batterham ym. \cite{BatterhamDepressiveDisorder} 

\subsection{Silmätautidiagnostiikan päätöksentuki}
Kabari ja Nwachukwu \cite{EyeDiseaseDiagnosis} kehittivät silmätautien diagnostiikkaan päätöksentuen menetelmiä,
jossa ensin neuroverkkoa käytettiin diagnostiikkaan liittyvien luokitteluongelmien ratkaisuun ja tämän jälkeen
päätöspuuta luokittelun saattamiseksi symboliseen, ymmärrettävään muotoon. Tällaisessa asetelmassa neuroverkko
ensin koulutetaan ja testataan tavalliseen tapaan opetus- ja testijoukoilla, jonka jälkeen kyseistä yleistämiseen
kykenevää neuroverkkoa käytetään uusien opetus- ja testijoukkojen muodostamiseen. Tämä tapahtuu yksinkertaisesti
syöttämällä neuroverkolle uusia tapauksia ja tallentamalla neuroverkon aikaansaama luokitus. Tätä uutta dataa
käytetään sitten päätöspuun muodostamisessa ja testaamisessa. Tässä tapauksessa päätöspuun induktiossa käytettiin
Quinlanin C4.5-algoritmia.



\section{Yhteenveto}
Päätöspuut ovat tehokkaita luokittimia. Luokittelijoina päätöspuut ovat siitä erikoisia, että
tapa, jolla ne aikaansaavat luokittelun, on intuitiivisesti suhteellisen helppo kenen tahansa
ymmärtää. Puiden rakenteesta, sen solmuista, lehdistä ja poluista on parhaimmassa tapauksessa
aikaansaatavissa sellaisia, että ne perusteet, jolla päätöspuu aikaansaa luokittelun, ovat
myös kenen tahansa ymmärrettävissä. Tämän vuoksi on oletettavissa, että erilaisista luokittimista
juuri päätöspuut saattaisivat soveltua sellaisiin tilanteisiin, jossa päätöspuuta käyttävän
pitäisi kyetä varmistumaan, että sen tulos on järkeenkäypä. Etenkin terveydenhuollon
päätöksentekotilaneissa, joissa päätöspuita voitaisiin soveltaa päätöksenteon tukemiseen,
tällainen verifiointimahdollisuus saattaisi olla juuri se ratkaiseva tekijä, jolla 
terveydenhuollon ammattilaisten olisi helpompi hyväksyä koneellisia apukeinoja hankalien
päätöksien tekemiseen. Esimerkkinä terveydenhuollon sovelluksista nostettiin päätöspuu,
jota voidaan käyttää arvioitaessa potilaiden riskiä sairastua aivoverenkiertohäiriöihin. 
Puu osoittauti selkeäksi ja sen logiikka järkeenkäyväksi. Muutama lääkäri oli todennut
puun alustavasti käyttökelpoiseksi, kuitenkaan puuta ei oltu kokeiltu terveydenhuollon
päivittäisessä toiminnassa, mikä olisi luonnollisesti oleellista se selvittämiseksi,
onko päätöspuihin perustuvilla luokittimilla ja päätöksenteon tukimenetelmillä
käyttökelpoisuutta terveydenhuollon kontekstissa.


% --- References ---
%
% bibtex is used to generate the bibliography. The babplain style
% will generate numeric references (e.g. [1]) appropriate for theoretical
% computer science. If you need alphanumeric references (e.g [Tur90]), use
%
% \bibliographystyle{babalpha-lf}
%
% instead.

\bibliographystyle{babplain-lf}
\bibliography{references-fi}


% --- Appendices ---

% uncomment the following

% \newpage
% \appendix
% 
% \section{Esimerkkiliite}

\end{document}
